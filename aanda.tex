%                                                                 aa.dem
% AA vers. 8.2, LaTeX class for Astronomy & Astrophysics
% demonstration file
%                                                       (c) EDP Sciences
%-----------------------------------------------------------------------
%
%\documentclass[referee]{aa} % for a referee version
%\documentclass[onecolumn]{aa} % for a paper on 1 column  
%\documentclass[longauth]{aa} % for the long lists of affiliations 
%\documentclass[rnote]{aa} % for the research notes
%\documentclass[letter]{aa} % for the letters 
%\documentclass[bibyear]{aa} % if the references are not structured 
% according to the author-year natbib style

%
\documentclass{aa}  

%
\usepackage{graphicx}
%%%%%%%%%%%%%%%%%%%%%%%%%%%%%%%%%%%%%%%%
\usepackage{txfonts}
%%%%%%%%%%%%%%%%%%%%%%%%%%%%%%%%%%%%%%%%
%\usepackage[options]{hyperref}
% To add links in your PDF file, use the package "hyperref"
% with options according to your LaTeX or PDFLaTeX drivers.
%
\begin{document} 


   \title{Using ExPRES for the ESA-JUICE Mission Planning}

   \subtitle{}

   \author{B. Cecconi
          \inst{1}
          \and
          C. Louis\inst{2}
          }

   \institute{LESIA, Observatoire de Paris, CNRS, PSL Research University, Meudon, France\\
              \email{baptiste.Cecconi@observatoiredeparis.psl.eu}
         \and
             IRAP, CNRS, Université Paul Sabatier, Toulouse, France\\
             \email{corentin.louis@irap.omp.eu}
             }

   \date{}

% \abstract{}{}{}{}{} 
% 5 {} token are mandatory
 
  \abstract
  % context heading (optional)
  % {} leave it empty if necessary  
   {}
  % aims heading (mandatory)
   {}
  % methods heading (mandatory)
   {}
  % results heading (mandatory)
   {}
  % conclusions heading (optional), leave it empty if necessary 
   {}

   \keywords{Radio emissions --
                Jupiter -- 
                JUICE mission
               }

   \maketitle
%
%________________________________________________________________

\section{Introduction}

   

%__________________________________________________________________

\section{Galileo Flyby}
We test the ExPRES prediction capabilities on the first flyby of Ganymede by the Galileo. 



\end{document}
